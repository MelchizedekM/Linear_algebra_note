\documentclass{report}
\usepackage{xeCJK}
\usepackage{amssymb}
\usepackage{babel}
\usepackage{geometry}
\usepackage{amsmath}
\usepackage{amsthm}
\usepackage{framed}
\usepackage{pifont}
\usepackage{listings}
\usepackage{tcolorbox}
\usepackage{blindtext}
\usepackage{xcolor}
\usepackage{color}
\usepackage{chngcntr}

\counterwithin{table}{chapter}


\def\ra{\rightarrow}
\def\rr{\Rightarrow}
\def\tf{$\Rightarrow$}
\def\oo{\infty}
\def\l/{\backslash}
\def\0{\emptyset}
\def\px{\mathcal{P}_X}
\def\s{\mathcal{S}}
\def\a{\mathcal{A}}
\def\bs{\mathcal{B}}
\def\lm{\mathcal{L}}
\def\R{\mathbb{R}}
\def\E{\mathbb{E}}
\def\Z{\mathbb{Z}}
\def\m{\mathcal{M}}
\def\b{\,\,\,}
\def\mm{{\mu^*}}
\def\hm{\mathcal{H}^s}
\def\vm{\nu^*}
\def\cui{\bigcup_{i=1}^\infty}
\def\cai{\bigcap_{i=1}^\infty}
\def\cuj{\bigcup_{j=1}^\infty}
\def\caj{\bigcap_{j=1}^\infty}
\def\sumj{\sum_{j=1}^\infty}
\def\sumi{\sum_{i=1}^\infty}
\def\sumn{\sum_{n=1}^\infty}
\def\sums{\sum_{y \in \s}}
\def\f{\mathcal{F}}
\def\limn{\lim_{n \rightarrow \infty}}
\def\st{\text{s.t.}}
\def\sums{ \sum_{x \in \s}}
\def\baru{\bar{\mu}}
\def\Mbaru{\mathfrak{M}_{\baru}}
\def\V{\mathcal{V}}
\def\U{\mathcal{U}}
\def\v{\mathbf{v}}
\def\u{\mathbf{u}}
\def\w{\mathbf{w}}
\def\span{\text{span}}
\def\<{\left\langle}
\def\>{\right\rangle}
\def\¥{$}
\def\¥{$}


\newtheorem{definition}{Definition}[chapter]
\newcommand{\deftitle}[1]{\begin{definition}  #1 \end{definition}}
\newtheorem{property}{Property}[chapter]
\newcommand{\proptitle}[1]{\begin{property}  #1 \end{property}}

\newcommand{\defb}[1]{\begin{tcolorbox} [colback=orange!20, colframe=orange!60, sharp corners, leftrule={3pt}, rightrule={0pt}, toprule={0pt}, bottomrule={0pt}, left={2pt}, right={2pt}, top={3pt}, bottom={3pt}] \deftitle{#1}} 
\def\defe{\end{tcolorbox}}
\newcommand{\propb}[1]{\begin{tcolorbox} [colback=red!10, colframe=red!60, sharp corners, leftrule={3pt}, rightrule={0pt}, toprule={0pt}, bottomrule={0pt}, left={2pt}, right={2pt}, top={3pt}, bottom={3pt}]\proptitle{#1}}
\def\prope{\end{tcolorbox}}
\def\egb{\begin{tcolorbox} [colback=cyan!20, colframe=cyan!40, sharp corners, leftrule={5pt}, rightrule={0pt}, toprule={0pt}, bottomrule={0pt}, left={2pt}, right={2pt}, top={3pt}, bottom={3pt}]}
\def\ege{\end{tcolorbox}}




\title{线性代数笔记}

\author{M.}
\date{From Oct. 12 2023 \\ To \today}

\begin{document}

	

\maketitle

\newpage

\tableofcontents



\chapter{Vector, Vector space, and Linear}

欢迎来到线性代数课程的第一章,本章会从之前所学的对向量的朴素认知出发发展出什么是vector space,进而延伸到对linear的思考和定义。

\section{Vector and Vector space}

在高中的学习过程中,已经接触到了一种特殊的量——向量(或矢量)。在过去的学习中认为这种量不只有数值大小,还有方向。一般会将他们写成(a,b,c)的形式。例如(6,1,2)就代表着一个在x方向延伸六个单位长度,在y方向延伸一个单位长度,在z方向延伸了两个单位长度的向量。但是这种表述方式并不够规范,也不好表示一组有着某些抽象关系的向量,例如所有从原点发出的向量。毕竟像 $ \{(x,y,z):x\in  \mathbb{R},y\in \mathbb{R},z \in \mathbb{R} \} $ 的表达并不算简单,在引入更多坐标轴变量的时候也会更费墨水,因此想要引入一些特殊的符号来更加方便的表述这样的list(有序列)。

首先严谨的定义一下什么是list和list的length。
\defb{list}
	定义n个数构成了一个length为n的list当且仅当这n个数以一种有序的方式排列在一起组合成一个元素。
	一般将这种有序列写作 $ (x_1,x_2,x_3,…,x_n) $ 。
\defe


在此基础上,便可以定义在任意n下所有长度为n的list的集合了
定义 $ \mathbb{R}^n $ 为所有长度为n的list的集合,即

\defb{$\R^n$} 
 $$ \mathbb{R}^n := \{(x_1,x_2,x_3,…,x_n):\forall i\in \mathbb{Z} \,\, x_i\in  \mathbb{R} \} $$ 
称其中的 $ x_i $ 为the i-th coordinate。
\defe

\egb
例如:
 $$ \mathbb{R}^4 := \{(x_1,x_2,x_3,x_4):\forall i\in \mathbb{Z} \,\, x_i\in  \mathbb{R} \} $$ 
\ege

现在非常神奇地发现,之前所学过的所谓vector都是属于 $\mathbb{R}^2$ 或 $\mathbb{R}^3$ 的list。回忆曾经学过的向量的加法,比如(6,1,2)+(12,1,8)=(6+12,1+1,2+8)加法的方式就是简单的将不同coordinate上的元素分别相加。

那么类比的,就可以定义任意n下 $\mathbb{R}^n$ 中元素的加法:

\defb{加法}
	在 $\mathbb{R}^n$ 中符号“+”可以由以下公式定义,将这种运算称为\textbf{加法}:	
 $$ \forall x,y\in \mathbb{R}^n,x+y∶=z \,\,\, \text{where} \,\,\, z_i=x_i+y_i  \forall i $$ 
\defe

可以显然的发现,这里的加法是符合交换律的,在这里就不做证明,留待读者自证。而根据对零的一般感知,也就是所有数加0都是它本身,可以定义在 $\mathbb{R}^n$ 中的0:

\defb{0}
	定义 $\mathbb{R}^n$ 中的\textbf{0}为:
 $$ \forall x\in \mathbb{R}^n \,\,\,  x+0=x $$ 
\defe
可以显然的发现在所有 $\mathbb{R}^n$ 中\textbf{0都是唯一的并且0=(0,0,0,…,0)}.

当然,数学家不愿意止步于此。希望将熟悉的欧几里得空间也就是 $\mathbb{R}^n$ 中的朴素认知拓展推广到更加抽象的一般集合中。再根据对vector也就是向量的朴素认知将其推广到一般集合中的一般元素。便有了更加抽象和一般化的对vector space和vector的定义:

\defb{vector space}
定义一个集合 $ \mathcal{V} $ 是一个\textbf{vector space}当且仅当它满足以下所有性质:

	\begin{itemize}
	
	\item[1.] Commutativity:  $\forall  \u,\v\in \mathcal{V} \,\, \u+\v=\v+\u$ 

	\item[2.] Associativity: $ \forall  \u,\v,\w \in \mathcal{V} \,\, \u+(v+w)=(\v+\u)+\w \,\,\, \text{and} \,\,\, \forall  a,b\in \mathbb{R} \,\, a(b\v)=(ab)\v $ 
	
	\item[3.] Additive identity: $ \exists \u \in \mathcal{V} \,\, \forall  \v\in \mathcal{V} \,\, \v+\u=\v \b \text{将} \u \text{记作} 0 $ 

	\item[4.] Additive inverse: $ \forall  \v\in \mathcal{V} \,\, \exists  \u\in \mathcal{V} \,\, \v+\u=0 $ 
	
	\item[5.] Multiplicative identity: $ \forall  \v\in \mathcal{V} \,\, 1\v=\v $ 
	
	\item[6.] Distributive properties: $ \forall  \u,\v\in \mathcal{V} \,\, \forall  a,b\in \mathbb{R} \,\,  a(\v+\u)=a\v+a\u \,\, (a+b)\v=a\v \,\,\,  \text{and} \,\,\,  +b\v $ 
	\end{itemize}
同时定义所有属于 $ \mathcal{V} $ 这个vector space的元素都是 \textbf{vector}。
\defe

在这个定义中,默认了数乘,也就是 $ \forall a \in \R \b \forall \v \in \V \b a\v \in \V $,的存在性。这也指出当说一个集合$\V$是一个vector space之前,必须先定义这个集合中元素的数乘。

\egb
例如:

$\R^n$之中,$\forall a \in \R \b \forall \v = (v_1,v_2,\dots,v_n) \in \R^n$ 定义 $ a\v = (av_1,av_2,\dots,av_n) $为$\R$中的数乘。

这和在高中阶段所学的向量的数乘也是一致的。
\ege

可以很简单的验证 $ \mathbb{R}^n $ 是一个vector space。

\egb
而也可以举出一些更加奇怪的vector space的例子:

如果定义 $ P(n) ∶=\{ \sum_{i=0}^n p_i t^i :\forall p_i\in \mathbb{R}\} $  那么可以发现P(n)也是一个vector space。
\ege 

自然的,我们会思考,像P(n)和$\mathbb{R}^n$的这些vector space之间会不会有一种内在联系呢?
显然会发现任意一个 $ \sum_{i=0}^n p_i t^i $ 的多项式都可以被 $ (p_1,p_2,…,p_n) $ 唯一确定的表示,将这种关系称为\textbf{isomorphism}。对于这一关系的定义完善先按下不表,继续对于Vector space本身的探讨。

\section{Linear, Span and Subspace}

在上一个小节,介绍了什么是vector space。而在这一章节中,将会从一个重要的概念——linear出发,探讨该如何构建一个vector space以及vector的一些重要定义。

当提及linear也就线性的时候,第一时间想到的一定会是线性函数(linear function),也就是从初中开始接触到的一次函数。根据对线性函数的直观印象,可以简单地认为,线性本质上就是对不同的元素进行加法和数乘的运算。而在上一章节中已经定义了vector space,也就是说,可以将线性函数初中学习的最基础的多元一次函数拓展到vector space上,看作是对vector space中的元素进行加法和数乘的运算。而将这种对vector space中的多个元素进行加法和数乘的运算称为\textbf{linear combination}。

\defb{linear combination}
	$\forall \v_1,\v_2,…,\v_k \in \V \b \forall a_1,a_2,…,a_k \in \R $ 称形如$ \sum_{i=1}^{k} a_k \v_k $的表达式为$\v_1, \v_2 , \dots , \v_k$的一个\textbf{linear combination}。
	
	如果$ \u \in \V$ 是由通过关系式 $\u = a_1\v_1+a_2\v_2+…+a_n\v_n $ 获得的,那么说$\u$可以被$ \v_1,\v_2,…,\v_n $ 的\textbf{linear combination}表示。
	
\defe

\egb
	例如:

	可以说 $ (6,1,2) $ 可以被 $ (1,0,0),(0,1,0),(0,0,1) $ 的linear combination表示,因为 $ (6,1,2) = 1(1,0,0)+1(0,1,0)+2(0,0,1) $。
\ege

这样的,我们对数学中linearity(线性)这一概念感知也就不言自明:\textbf{当某一种元素(可以是vector,number也可以是function或map)在其作用方式下对加法具有交换和结合率、并且不在意数乘的先后顺序,我们便可以认为元素是线性的}。这种感知在vector中就表现为vector space中的Commutativity、Associativity和Distributive properties。那么更进一步,对于一个function或map,该如何定义线性呢?在这里,做如下符合这一线性感知的定义:

\defb{linear}
	定义一个function或map $ f \, : \a \ra \bs $是\textbf{linear}的当且仅当它满足以下所有性质:
	\begin{itemize}
		\item[1.] $ \forall \u,\v \in \a \b f(\u+\v)=f(\u)+f(\v) \in \bs$
		\item[2.] $ \forall \v \in \a \b \forall a \in \R \b f(a\v)=af(\v) \in \bs$
	\end{itemize}

	在这里,我们称$\a$为domain,$\bs$为range或codomain。
	
	另外,我们把所有从$\a$到$\bs$的linear function或map的集合记做$\lm(\a,\bs)$。
\defe

\egb
	例如:

	$ f \, : \R^2 \ra \R^2 $ 定义为 $ f((x,y)) = (x+y,x-y) $ 是linear的,因为 $ f((x_1,y_1)+(x_2,y_2)) = f((x_1+x_2,y_1+y_2)) = (x_1+x_2+y_1+y_2,x_1+x_2-y_1-y_2) = (x_1+y_1,x_1-y_1)+(x_2+y_2,x_2-y_2) = f((x_1,y_1))+f((x_2,y_2)) $ 并且 $ f(a(x,y)) = f((ax,ay)) = (ax+ay,ax-ay) = a(x+y,x-y) = af((x,y)) $。
\ege

如果将vector space的定义视作构建大楼的图纸,将vector是为构建大楼的砖块。那么linear combination就可以被视为搭建蓝图中大厦所必要的工具(这里必须指出,虽然在定义什么是vector的时候使用了vector space。但是事实上这并不是必须的,甚至在许多情况下,是通过对一组感兴趣的元素的线性叠加来构建出所希望研究的vector space的。值得指出的是,这种循环看起来会产生形式上的问题。但在形式上依旧可以说“通过某一组‘元素’构建起了一个集合,而这个集合经过验证‘居然神奇的’是一个vector space。而最初用来构建这一vector space的元素‘居然’也是这个vector space中的元素”这样的过程来避免形式上的问题。在之后不会再注意使用这种冗长的避免方式,而是直接将构建的元素称为vector)。因此,引入一个重要的定义——\textbf{span},来严格表示使用一组vector来生成一个vector space的过程。

\defb{span}
	定义一个集合$\U$是由一组vector $\v_1,\v_2,\dots,\v_n \in \V$ \textbf{“span”}的当且仅当集合$\U$中的所有元素恰好是$\v_1,\v_2,\dots,\v_n$的全部linear combination。将这样的$\U$记做$\span(\v_1,\v_2,\dots,\v_n)$。

	用一种更加数学的表示方法,可以将$\span(\v_1,\v_2,\dots,\v_n)$表示为:

	$$ \span(\v_1,\v_2,\dots,\v_n) = \left\{ \sum_{i=1}^{n} a_k \v_k : \forall a_k \in \R \right\} $$
\defe

\egb
例如:

\begin{itemize}

\item  $(6,0,0)$和$(0,12,0)$的span,即$\span((6,0,0),(0,12,0))$是在$\R^3$中的一个平面,即所有$z=0$的点的集合。


\item $f_1(t) = t^2$和$f_2(t) = t^3$的span,即$\span(f_1,f_2)$是在$P(3)$中的一个子空间,即所有次数只有二次项和三次项的多项式的集合。

\end{itemize}

\ege 

在这个例子中,不难发现,$\span((6,0,0),(0,12,0))$的结果也是一个vector space。这个vector space是属于$\R^3$的个子集。而$\span(f_1,f_2)$的结果也是一个vector space。这个vector space是属于$P(3)$的一个子集。那么更一般化的,我们希望将这样的子空间定义化——称为\textbf{subspace}。

\defb{subspace} 
	定义一个集合$\U$是$\V$的一个\textbf{subspace}当且仅当它满足以下所有性质:
	\begin{itemize}
		\item[1.] $\U$是一个vector space
		\item[2.] $\U \subseteq \V$
	\end{itemize}

\defe 

在获得了subspace的定义后,可以发现任意一个vector space都拥有大于等于一个的subspace。基于集合的性质,我们自认会考虑不同subspace之间加和。在这里我们引入两种加和方式——\textbf{sum}和\textbf{direct sum}。

\defb{sum} 
	定义两个subspace $\U$和$\V$的\textbf{sum}(记做$+$)为:
	$$ \U + \V := \{ \u + \v : \forall \u \in \U \b \forall \v \in \V \} $$
\defe

\defb{direct sum} 
	定义两个subspace $\U$和$\V$的sum是\textbf{direct sum}(记做$\oplus$)为:
	$$ \U \oplus \V := \{ \u + \v : \forall \u \in \U \b \forall \v \in \V \b \u + \v \text{是唯一的} \} $$
\defe

\egb
例如:

\begin{itemize}

\item  $\span({6,0,0},(0,12,0))$和$\span((12,0,0),(0,18,0))$的sum,即$\span((6,0,0),(0,12,0))+\span((12,0,0),(0,18,0))$,是在$\R^3$中\¥ z = 0 \¥ 的平面。由于\¥ (18,31,0) \¥ 可以表示为\\$(18,31,0) + 0 \b (\text{where }(18,31,0)\in \span((6,0,0),(0,12,0))), \, 0 \in \span((12,0,0),(0,18,0))$\\或$(6,12,0)+(12,18,0) (\text{where } (6,12,0) \in \span((6,0,0),(0,12,0))), \, (12,18,0) \in \span((12,0,0),(0,18,0))$。\\ 这说明$\span({6,0,0},(0,12,0))$和$\span((12,0,0),(0,18,0))$的sum并不少direct sum。
\item $\span({6,0,0},(0,12,0))$和$\span((0,0,1))$的sum,我们可以容易验证这二者的sum是一个direct sum。在此不加赘述,留待读者自证。

\end{itemize}
\ege
\section{Matrix and Matrix computation}

回顾在讲解$\span$的过程中距离使用的的线性组合。可以发现,这种线性组合可以被表示为一组多元一次方程组的形式。可以将任意的$\v \in \span((6,0,0),(0,12,0),(0,0,1))$表示为:

$$\begin{cases}
	6x_1 + 0x_2 + 0x_3 &= v_1 \\
	0x_1 + 12x_2 + 0x_3 &= v_2 \\
	0x_1 + 0x_2 + 1x_3 &= v_3
\end{cases}$$

让在进一步观察这个线性方程组的形式,可以发现如果放弃将vector写成横向,而是以竖向表示,那么
$ \left[\begin{matrix}
	v_1 \\
	v_2 \\
	v_3
\end{matrix} \right]$ 
=
$ \left[\begin{matrix}
	6x_1 + 0x_2 + 0x_3\\
	0x_1 + 12x_2 + 0x_3\\
	0x_1 + 0x_2 + 1x_3
\end{matrix} \right]$ 
。

可以发现,$x_1,x_2,x_3$的出现是存在规律的,在每一行上第一个出现的都是$x_1$第二个都是$x_2$...以此类推。而且不难验证,这种关系对于任意的$n$都是普遍存在的。作为懒惰且不想多费墨水的,自然会想要寻找一种更加简单的方式来反应这样的组合。由于$x_i$的出现位置是一定的,因而只有每个$x_i$前面的系数是重要的。需要做的只是将这些系数按照一定的顺序保存下来,而这种保存的方式就是\textbf{matrix}。

\defb{matrix} 
	定义一个 $ m \times n $ 的形如
	$\left[ \begin{matrix}
		a_{11} & a_{12} & \dots & a_{1n} \\
		a_{21} & a_{22} & \dots & a_{2n} \\
		\vdots & \vdots & \ddots & \vdots \\
		a_{m1} & a_{m2} & \dots & a_{mn}		
	\end{matrix}\right]$
	的二维自然数阵列($\forall i \in [1,m] \, j \in [1,n] \b a_{ij} \in \R $)称为一个\textbf{matrix}。一般使用大写字母的粗体字母\textbf{A,B,C}\dots 来表示矩阵。\\
	
	将矩阵的第$i$行第$j$列的元素称为矩阵的第$i$行第$j$列的\textbf{entry}。将矩阵的第$i$行的所有entry组成的list称为矩阵的第$i$行的\textbf{row}。将矩阵的第$j$行的所有entry组成的list称为矩阵的第$j$列的\textbf{column}。\\

	特别的,将这种反应线性方程系数关系的matrix称为\textbf{coefficient matrix}。
\defe

\egb
例如:

对于刚刚提到的例子,
$\left[ \begin{matrix}
	6 & 0 & 0 \\
	0 & 12 & 0 \\
	0 & 0 & 1
\end{matrix}\right]$
就是这个线性方程组的coefficient matrix。
\ege

在这里,我们不难发现vector本质上就是一个1column的matrix后,\textbf{matrix sum}和\textbf{scalar multiplication in matrix}便可以根据我们对vector sum和vector scalar multiplication的认识自然产生了:

\defb{matrix sum}
定义\textbf{C}是\textbf{A}和\textbf{B}的\textbf{matrix sum}当且仅当\textbf{A}和\textbf{B}的行数和列数分别相等并且$\forall i,j \in \R \b c_{ij}=a_{ij}+b_{ij}$。

记做\textbf{C}=\textbf{A}+\textbf{B}。
\defe

\defb{scalar multiplication in matrix}
定义\textbf{B}是\textbf{A}的\textbf{scalar multiplication in matrix}当且仅当$\forall i,j \in \R \b \text{for some } \alpha \b b_{ij}= \alpha a_{ij}$。

记做\textbf{B}=$\alpha$\textbf{A}。
\defe

\egb
例如:

$\left[ \begin{matrix}
	6 & 0 & 0 \\
	0 & 12 & 0 \\
	0 & 0 & 1
\end{matrix}\right]$
+
$\left[ \begin{matrix}
	1 & 0 & 0 \\
	0 & 1 & 0 \\
	0 & 0 & 1
\end{matrix}\right]$
=
$\left[ \begin{matrix}
	7 & 0 & 0 \\
	0 & 13 & 0 \\
	0 & 0 & 2
\end{matrix}\right]$

$2\left[ \begin{matrix}
	6 & 0 & 0 \\
	0 & 12 & 0 \\
	0 & 0 & 1
\end{matrix}\right]$
=
$\left[ \begin{matrix}
	12 & 0 & 0 \\
	0 & 24 & 0 \\
	0 & 0 & 2
\end{matrix}\right]$
\ege

不难发现,对于所有的$n$和$m$

进一步,有没有可能定义出一种运算,来直接反应$x_1,x_2,x_3$是如何做用到这个coefficient matrix上的呢?答案是肯定的。由于需要将$x_1,x_2,x_3$的这种有序关系保留下来,会自然的想到将$x_1,x_2,x_3$表示为一个vector,即
$\mathbf{x} = 
\left[ \begin{matrix}
	x_1 \\
	x_2 \\
	x_3
\end{matrix}\right]
$
。定义运算 
$
\left[ \begin{matrix}
	6 & 0 & 0 \\
	0 & 12 & 0 \\
	0 & 0 & 1
\end{matrix}\right]
\left[ \begin{matrix}
	x_1 \\
	x_2 \\
	x_3
\end{matrix}\right]
=
\left[ \begin{matrix}
	6x_1 + 0x_2 + 0x_3\\
	0x_1 + 12x_2 + 0x_3\\
	0x_1 + 0x_2 + 1x_3
\end{matrix}\right]
$。
将这种运算称为\textbf{matrix-vector multiplication}。

\defb{matrix-vector multiplication} 
	定义 $ m \times n $ 的matrix \textbf{A}和一个 length 为$ n $ vector \textbf{x}的\textbf{matrix-vector multiplication},$ \mathbf{Ax} $, 为
	$$ \mathbf{Ax} =
	 \left[ \begin{matrix}
		a_{11} & a_{12} & \dots & a_{1n} \\
		a_{21} & a_{22} & \dots & a_{2n} \\
		\vdots & \vdots & \ddots & \vdots \\
		a_{m1} & a_{m2} & \dots & a_{mn}
	\end{matrix}\right]
	\left[ \begin{matrix}
		x_1 \\
		x_2 \\
		\vdots \\
		x_n
	\end{matrix}\right]
	=
	\left[ \begin{matrix}
		a_{11}x_1 + a_{12}x_2 + \dots + a_{1n}x_n \\
		a_{21}x_1 + a_{22}x_2 + \dots + a_{2n}x_n \\
		\vdots \\
		a_{m1}x_1 + a_{m2}x_2 + \dots + a_{mn}x_n
	\end{matrix}\right]
	=: \mathbf{b}
	$$
	\\

	不难发现,其中$ \mathbf{b} $是一个length为$ m $的vector,它的第i个元素$ b_{i} $为A的第$i$行和x的第$j$列的“点乘”(对于这一高中知识将会在后面进行更加全面和数学化的介绍,其名称为 \textbf{inner product}。不同于高中$\v \cdot \u $的记法,一般会将这种运算写为$\< \v , \u \> $。在此后如果继续提及点乘,可能会将两种写法混用)。

\defe

更进一步的,是否可以同时计算多个用同一个coefficient matrix表示的线性方程组呢?答案是肯定的。只需要将多个vector按照一定的顺序排列在一起,形成一个matrix,然后将这个matrix和coefficient matrix进行matrix-vector multiplication即可。这样,就引入了一个新的概念——\textbf{matrix multiplication}。

\defb{matrix multiplication} 
	定义一个$ m \times n $ 的matrix \textbf{A}和一个$ n \times k $的matrix \textbf{B}的\textbf{matrix multiplication},$ \mathbf{AB} $, 为
	$$ \mathbf{AB} =
	 \left[ \begin{matrix}
		a_{11} & a_{12} & \dots & a_{1n} \\
		a_{21} & a_{22} & \dots & a_{2n} \\
		\vdots & \vdots & \ddots & \vdots \\
		a_{m1} & a_{m2} & \dots & a_{mn}
	\end{matrix}\right]
	\left[ \begin{matrix}
		b_{11} & b_{12} & \dots & b_{1k} \\
		b_{21} & b_{22} & \dots & b_{2k} \\
		\vdots & \vdots & \ddots & \vdots \\
		b_{n1} & b_{n2} & \dots & b_{nk}
	\end{matrix}\right]$$
	$$=
	\left[ \begin{matrix}
		a_{11}b_{11} + a_{12}b_{21} + \dots + a_{1n}b_{n1} & \dots & a_{11}b_{1k} + a_{12}b_{2k} + \dots + a_{1n}b_{nk} \\
		a_{21}b_{11} + a_{22}b_{21} + \dots + a_{2n}b_{n1} & \dots & a_{21}b_{1k} + a_{22}b_{2k} + \dots + a_{2n}b_{nk} \\
		\vdots  & \ddots & \vdots \\
		a_{m1}b_{11} + a_{m2}b_{21} + \dots + a_{mn}b_{n1} & \dots & a_{m1}b_{1k} + a_{m2}b_{2k} + \dots + a_{mn}b_{nk}
	\end{matrix}\right]
	=: \mathbf{C}
	$$
	\\

	不难发现,其中$ \mathbf{C} $是一个$ m \times k $的matrix,它的第$i$行和第$j$列的entry为A的第$i$行和B的第$j$列的inner product。

\defe 

\egb
例如:

\begin{itemize}

	\item 
	$\left[ \begin{matrix}
		6 & 0 & 0 \\
		0 & 12 & 0 \\
		0 & 0 & 1
	\end{matrix}\right]
	\left[ \begin{matrix}
		12  \\
		1  \\
		8 
	\end{matrix}\right]
	=
	\left[ \begin{matrix}
		6 \cdot 12 + 0 \cdot 1 + 0 \cdot 8 \\
		0 \cdot 12 + 12 \cdot 1 + 0 \cdot 8 \\
		0 \cdot 12 + 0 \cdot 1 + 1 \cdot 8
	\end{matrix}\right]
	=
	\left[ \begin{matrix}
		72 \\
		12 \\
		8
	\end{matrix}\right]
	$

	\item
	$\left[ \begin{matrix}
		6 & 0 & 0 \\
		0 & 12 & 0 \\
		0 & 0 & 1
	\end{matrix}\right]
	\left[ \begin{matrix}
		12 & 1 & 6 \\
		1 & 0 & 2 \\
		8 & 7 & 6
	\end{matrix}\right]
	=
	\left[ \begin{matrix}
		72 & 6 & 36 \\
		12 & 0 & 24 \\
		8 & 7 & 6
	\end{matrix}\right]
	$

\end{itemize}

\ege

观察matrix multiplication的定义,可以发现,matrix multiplication的运算顺序是不能交换的。也就是说,$ \mathbf{AB} \neq \mathbf{BA} $。这是因为matrix multiplication的定义中,matrix \textbf{A}的行数和matrix \textbf{B}的列数是有关系的。

并且matrix multiplication可以被视作是一种\textbf{composition}。也就是说,$ \mathbf{AB} $是将\textbf{A}作用在\textbf{B}(在这里\textbf{B}可以是一个只有一列的matrix,即vector)上的结果。在这种基础上,我们不难验证,其实所有的matrix multiplication都可以被视作从一个vector space到另一个vector space的linear function。即$\mathbf{A}(*) : \R^{n+k} \ra \R^{m+k}$,在这里\textbf{A}是一个m行n列的矩阵,*是一个n行k列的vector。

\propb{}
	对于任意一个$ n \times m $ 的matrix \textbf{A},其都可以唯一确定的表示一个linear function $\mathbf{A}(*) : \R^{n+k} \ra \R^{m+k}$。
	其中k是方程$\mathbf{A}(*)$作用到的matrix的列数。
\prope

\egb
例如:

我们上面提到的function $ f \, : \R^2 \ra \R^2 $ 定义为 $ f((x,y)) = (x+y,x-y) $,可以被表示为一个coefficient matrix $ \left[ \begin{matrix}
	1 & 1 \\
	1 & -1
\end{matrix}\right] $。那么我们可以发现,$ f((x,y)) = \left[ \begin{matrix}
	1 & 1 \\
	1 & -1
\end{matrix}\right] \left[ \begin{matrix}
	x \\
	y
\end{matrix}\right] $。这也就是说,$ f((x,y)) $可以被视作是将$ \left[ \begin{matrix}
	1 & 1 \\
	1 & -1
\end{matrix}\right] $作用在$ \left[ \begin{matrix}
	x \\
	y
\end{matrix}\right] $上的结果。
\ege

\section{Linear Independence, bases and dimension}

在上一章节中,我们已经知道了如何将一个vector space表示为一组vector的span。同样的,我们也知道对于同样一个vector space,可以有多种不同的span(例如$\R^2$可以被表示为$\span((1,0),(0,1))$,也可以被表示为$\span((1,1),(-1,1),(1,0))$)。那么,这些能生成同一个vector space的span vector之间有什么联系呢?不难看出,对于$(1,1),(-1,1),(1,0)$这一组vector中任意一个都可以被其他两个的linear combination表示。这样的vector组合称为\textbf{linearly dependent}。而对于$(1,0),(0,1)$这一组vector中任意一个都不可以被其他一个的linear combination表示。这样的vector组合称为\textbf{linearly independent}。

\defb{linearly dependent}
	定义一组vector $\v_1,\v_2,\dots,\v_n \in \V$是\textbf{linearly dependent}当且仅当至少存在一组不全为0的$a_1,a_2,\dots,a_n \in \R$使得$\sum_{i=1}^{n} a_i \v_i = 0$。

	或者说,$\v_1,\v_2,\dots,\v_n \in \V$是\textbf{linearly dependent}当且仅当至少存在一个$i \in [1,n]$使得$\v_i \in \span(\v_1,\v_2,\dots,\v_{i-1},\v_{i+1},\dots,\v_n)$。
\defe

\defb{linearly independent}
	定义一组vector $\v_1,\v_2,\dots,\v_n \in \V$是\textbf{linearly independent}当且仅当$\v_1,\v_2,\dots,\v_n \in \V$不是\textbf{linearly dependent}。

	或者说,$\v_1,\v_2,\dots,\v_n \in \V$是\textbf{linearly independent}当且仅当$\v_1,\v_2,\dots,\v_n \in \V$不是\textbf{linearly dependent}。
\defe

在这个基础之上,我们可以定义出一组\textbf{span vector}是一个vecror spece $ \V$ 的\textbf{basis}。

\defb{basis}
	定义一组vector $\v_1,\v_2,\dots,\v_n \in \V$是\textbf{basis}当且仅当$\v_1,\v_2,\dots,\v_n \in \V$是linearly independent并且$\span(\v_1,\v_2,\dots,\v_n) = \V$。

	或者说,$\v_1,\v_2,\dots,\v_n \in \V$是\textbf{basis}当且仅当$\v_1,\v_2,\dots,\v_n \in \V$是能够span $\V$的最小的span set。
\defe

而基于\textbf{span}的定义,不难知道如果$\v_1,\v_2,\dots,\v_n \in \V$是$\V$的\textbf{basis},那么对于任意的$\v \in \V$都可以被表示为$\v = \sum_{i=1}^{n} a_i \v_i$。再由于\textbf{bases}的性质,不难发现$a_1,a_2,\dots,a_n$是唯一的。

\propb{Criterion for basis}

$\v_1,\v_2,\dots,\v_n \in \V$是$\V$的\textbf{basis}当且仅当对于任意的$\v \in \V$都会有唯一的$\v = \sum_{i=1}^{n} a_i \v_i$表示。

\prope

\begin{proof}

	1) $\Rightarrow$ :

	可以被表示:

	由于$\v_1,\v_2,\dots,\v_n \in \V$是$\V$的\textbf{span},因而对于任意的$\v \in \V$都会有$\v = \sum_{i=1}^{n} a_i \v_i$表示。
	
	唯一性:

	假设对于某一个$\v \in \V$可以被$\v = \sum_{i=1}^{n} a_i \v_i$和$\v = \sum_{i=1}^{n} b_i \v_i$。
	
	那么可以发现$\v - \v = \sum_{i=1}^{n} a_i \v_i - \sum_{i=1}^{n} b_i \v_i = \sum_{i=1}^{n} (a_i - b_i) \v_i = 0$。 并且至少存在一个$i$使得$a_i - b_i \neq 0$。

	由于$\v_1,\v_2,\dots,\v_n$是linear independent,因而$a_i - b_i = 0$。这与假设矛盾。

	2) $\Leftarrow$ :

	由于任意的$\v \in \V$,都有可以被表示为$\v = \sum_{i=1}^{n} a_i \v_i$,因而$\v_1,\v_2,\dots,\v_n \in \V$是$\V$的\textbf{span}。

	由于任意的$\v \in \V$,都有唯一的$\v = \sum_{i=1}^{n} a_i \v_i$,因而$\v_1,\v_2,\dots,\v_n \in \V$是$\V$的\textbf{linearly independent}。
\end{proof}

\end{document}