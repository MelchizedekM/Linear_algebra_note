\documentclass{report}
\usepackage{xeCJK}
\usepackage{amssymb}
\usepackage{babel}
\usepackage{geometry}
\usepackage{amsmath}
\usepackage{amsthm}
\usepackage{framed}
\usepackage{pifont}
\usepackage{listings}
\usepackage{tcolorbox}
\usepackage{blindtext}



\def\ra{\rightarrow}
\def\rr{\Rightarrow}
\def\tf{$\Rightarrow$}
\def\oo{\infty}
\def\l/{\backslash}
\def\0{\emptyset}
\def\px{\mathcal{P}_X}
\def\s{\mathcal{S}}
\def\a{\mathcal{A}}
\def\bs{\mathcal{B}}
\def\lm{\mathcal{L}}
\def\R{\mathbb{R}}
\def\E{\mathbb{E}}
\def\Z{\mathbb{Z}}
\def\m{\mathcal{M}}
\def\b{\,\,\,}
\def\mm{{\mu^*}}
\def\hm{\mathcal{H}^s}
\def\vm{\nu^*}
\def\cui{\bigcup_{i=1}^\infty}
\def\cai{\bigcap_{i=1}^\infty}
\def\cuj{\bigcup_{j=1}^\infty}
\def\caj{\bigcap_{j=1}^\infty}
\def\sumj{\sum_{j=1}^\infty}
\def\sumi{\sum_{i=1}^\infty}
\def\sumn{\sum_{n=1}^\infty}
\def\sums{\sum_{y \in \s}}
\def\f{\mathcal{F}}
\def\limn{\lim_{n \rightarrow \infty}}
\def\st{\text{s.t.}}
\def\sums{ \sum_{x \in \s}}
\def\baru{\bar{\mu}}
\def\Mbaru{\mathfrak{M}_{\baru}}
\def\defb{\begin{tcolorbox} [colback=yellow!20, colframe=orange!80, sharp corners, leftrule={5pt}, rightrule={0pt}, toprule={0pt}, bottomrule={0pt}, left={2pt}, right={2pt}, top={3pt}, bottom={3pt}]}
\def\defe{\end{tcolorbox}}
\def\egb{\begin{tcolorbox} [colback=blue!20, colframe=blue!80, sharp corners, leftrule={5pt}, rightrule={0pt}, toprule={0pt}, bottomrule={0pt}, left={2pt}, right={2pt}, top={3pt}, bottom={3pt}]}
\def\ege{\end{tcolorbox}}
\def\V{\mathcal{V}}
\def\U{\mathcal{U}}
\def\v{\mathbf{v}}
\def\u{\mathbf{u}}
\def\w{\mathbf{w}}
\def\span{\text{span}}


\title{线性代数笔记}

\author{M.}
\date{From Oct. 12 2023 \\ To \today}

\begin{document}

\maketitle

\newpage

\tableofcontents

\chapter{Vector, Vector space, and Linear}

欢迎来到我们线性代数课程的第一章,本章我们会从之前所学的对向量的朴素认知出发发展出什么是vector space,进而延伸到对linear的思考和定义。

\section{Vector and Vector space}

在我们高中的学习过程中,我们已经接触到了一种特殊的量——向量(或矢量)。在过去的学习中我们认为这种量不只有数值大小,还有方向。我们一般会将他们写成(a,b,c)的形式。例如(6,1,2)就代表着一个在x方向延伸六个单位长度,在y方向延伸一个单位长度,在z方向延伸了两个单位长度的向量。但是这种表述方式并不够规范,我们也不好表示一组有着某些抽象关系的向量,例如所有从原点发出的向量。毕竟像 $ \{(x,y,z):x\in  \mathbb{R},y\in \mathbb{R},z \in \mathbb{R} \} $ 的表达并不算简单,在我们引入更多坐标轴变量的时候也会更费墨水,因此我们想要引入一些特殊的符号来更加方便的表述这样的list(有序列)。

我们首先严谨的定义一下什么是list和list的length。
\defb
	\textbf{Definition 1.1}
	我们说n个数构成了一个length为n的list当且仅当这n个数以一种有序的方式排列在一起组合成一个元素。
	我们一般将这种有序列写作 $ (x_1,x_2,x_3,…,x_n) $ 。
\defe


在此基础上,我们便可以定义在任意n下所有长度为n的list的集合了
我们定义 $ \mathbb{R}^n $ 为所有长度为n的list的集合,即

\defb 
	\textbf{Definition 1.2}	
 $$ \mathbb{R}^n := \{(x_1,x_2,x_3,…,x_n):\forall i\in \mathbb{Z} \,\, x_i\in  \mathbb{R} \} $$ 
称其中的 $ x_i $ 为the i-th coordinate。
\defe

\egb
例如:
 $$ \mathbb{R}^4 := \{(x_1,x_2,x_3,x_4):\forall i\in \mathbb{Z} \,\, x_i\in  \mathbb{R} \} $$ 
\ege

现在我们非常神奇地发现,我们之前所学过的所谓vector都是属于 $\mathbb{R}^2$ 或 $\mathbb{R}^3$ 的list。回忆我们曾经学过的向量的加法,比如(6,1,2)+(12,1,9)=(6+12,1+1,9+9)加法的方式就是简单的将不同coordinate上的元素分别相加。

那么类比的,我们就可以定义任意n下 $\mathbb{R}^n$ 中元素的加法:

\defb
	\textbf{Definition 1.3}
	在 $\mathbb{R}^n$ 中符号“+”可以由以下公式定义,我们将这种运算称为\textbf{加法}:	
 $$ \forall x,y\in \mathbb{R}^n,x+y∶=z \,\,\, \text{where} \,\,\, z_i=x_i+y_i  \forall i $$ 
\defe

我们可以显然的发现,这里的加法是符合交换律的,在这里我们就不做证明,留待读者自证。而根据我们对零的一般感知,也就是所有数加0都是它本身,我们可以定义在 $\mathbb{R}^n$ 中的0:

\defb
	\textbf{Definition 1.4}
	我们定义 $\mathbb{R}^n$ 中的\textbf{0}为:
 $$ \forall x\in \mathbb{R}^n \,\,\,  x+0=x $$ 
\defe
我们可以显然的发现在所有 $\mathbb{R}^n$ 中\textbf{0都是唯一的并且0=(0,0,0,…,0)}.

当然,我们的数学家不愿意止步于此。我们希望将我们熟悉的欧几里得空间也就是 $\mathbb{R}^n$ 中的朴素认知拓展推广到更加抽象的一般集合中。再根据我们对vector也就是向量的朴素认知将其推广到一般集合中的一般元素。我们便有了更加抽象和一般化的对vector space和vector的定义:

\defb
	\textbf{Definition 1.5}
我们说一个集合 $ \mathcal{V} $ 是一个\textbf{vector space}当且仅当它满足以下所有性质:

	1. Commutativity:  $\forall  \u,\v\in \mathcal{V} \,\, \u+\v=\v+\u$ 

	2. Associativity: $ \forall  \u,\v,\w \in \mathcal{V} \,\, \u+(v+w)=(\v+\u)+\w \,\,\, \text{and} \,\,\, \forall  a,b\in \mathbb{R} \,\, a(b\v)=(ab)\v $ 
	
	3. Additive identity: $ \exists \u \in \mathcal{V} \,\, \forall  \v\in \mathcal{V} \,\, \v+\u=\v \b \text{将} \u \text{记作} 0 $ 

	4. Additive inverse: $ \forall  \v\in \mathcal{V} \,\, \exists  \u\in \mathcal{V} \,\, \v+\u=0 $ 
	
	5. Multiplicative identity: $ \forall  \v\in \mathcal{V} \,\, 1\v=\v $ 
	
	6. Distributive properties: $ \forall  \u,\v\in \mathcal{V} \,\, \forall  a,b\in \mathbb{R} \,\,  a(\v+\u)=a\v+a\u \,\, (a+b)\v=a\v \,\,\,  \text{and} \,\,\,  +b\v $ 

同时定义所有属于 $ \mathcal{V} $ 这个vector space的元素都是 \textbf{vector}。
\defe

在这个定义中,我们默认了数乘,也就是 $ \forall a \in \R \b \forall \v \in \V \b a\v \in \V $,的存在性。这也指出当我们说一个集合$\V$是一个vector space之前,我们必须先定义这个集合中元素的数乘。

\egb
例如:

$\R^n$之中,$\forall a \in \R \b \forall \v = (v_1,v_2,\dots,v_n) \in \R^n$ 我们定义 $ a\v = (av_1,av_2,\dots,av_n) $为$\R$中的数乘。

这和我们在高中阶段所学的向量的数乘也是一致的。
\ege

我们可以很简单的验证 $ \mathbb{R}^n $ 是一个vector space。

\egb
而我们也可以举出一些更加奇怪的vector space的例子:

如果我们定义 $ P(n) ∶=\{ \sum_{i=0}^n p_i t^i :\forall p_i\in \mathbb{R}\} $  那么我们可以发现P(n)也是一个vector space。
\ege 

自然的,我们会思考,像P(n)和$\mathbb{R}^n$的这些vector space之间会不会有一种内在联系呢?
显然我们会发现任意一个 $ \sum_{i=0}^n p_i t^i $ 的多项式都可以被 $ (p_1,p_2,…,p_n) $ 唯一确定的表示,我们将这种关系称为\textbf{isomorphism}。对于这一关系的定义完善我们先按下不表,继续我们对于Vector space本身的探讨。

\section{Linear Combination and Span}

在上一个小节,我们介绍了什么是vector space。而在这一章节中,我们将会从一个重要的概念——linear出发,探讨我们该如何构建一个vector space以及vector的一些重要定义。

当我们提及linear也就线性的时候,我们第一时间想到的一定会是线性函数(linear function),也就是我们从初中开始接触到的一次函数。根据对线性函数的直观印象,我们可以简单地认为,线性本质上就是对不同的元素进行加法和数乘的运算。而我们在上一章节中已经定义了vector space,也就是说,我们可以将线性函数初中学习的最基础的多元一次函数拓展到vector space上,看作是对vector space中的元素进行加法和数乘的运算。而我们将这种对vector space中的多个元素进行加法和数乘的运算称为\textbf{linear combination}。

\defb
	\textbf{Definition 1.6}
	$\forall \v_1,\v_2,…,\v_k \in \V \b \forall a_1,a_2,…,a_k \in \R $ 我们称形如$ \sum_{i=1}^{k} a_k \v_k $的表达式为$\v_1, \v_2 , \dots , \v_k$的一个\textbf{linear combination}。
	
	如果$ \u \in \V$ 是由通过关系式 $\u = a_1\v_1+a_2\v_2+…+a_n\v_n $ 获得的,那么我们说$\u$可以被$ \v_1,\v_2,…,\v_n $ 的\textbf{linear combination}表示。
	
\defe

\egb
	例如:

	我们可以说 $ (6,1,2) $ 可以被 $ (1,0,0),(0,1,0),(0,0,1) $ 的linear combination表示,因为 $ (6,1,2) = 1(1,0,0)+1(0,1,0)+2(0,0,1) $。
\ege

如果我们将vector space的定义视作构建大楼的图纸,将vector是为构建大楼的砖块。那么linear combination就可以被视为搭建蓝图中大厦所必要的工具(这里我们必须指出,虽然我们在定义什么是vector的时候使用了vector space。但是事实上这并不是必须的,甚至在许多情况下,我们是通过对一组我们感兴趣的元素的线性叠加来构建出我们所希望研究的vector space的。值得指出的是,这种循环看起来会产生形式上的问题。但在形式上我们依旧可以说“我们通过某一组‘元素’构建起了一个集合,而这个集合经过验证‘居然神奇的’是一个vector space。而我们最初用来构建这一vector space的元素‘居然’也是这个vector space中的元素”这样的过程来避免形式上的问题。我们在之后不会再注意使用这种冗长的避免方式,而是直接将构建的元素称为vector)。因此,我们引入一个重要的定义——\textbf{span},来严格表示使用一组vector来生成一个vector space的过程。

\defb
	\textbf{Definition 1.7}
	我们称一个集合$\U$是由一组vector $\v_1,\v_2,\dots,\v_n \in \V$ \textbf{“span”}的当且仅当集合$\U$中的所有元素恰好是$\v_1,\v_2,\dots,\v_n$的全部linear combination。我们将这样的$\U$记做$\span(\v_1,\v_2,\dots,\v_n)$。

	用一种更加数学的表示方法,我们可以将$\span(\v_1,\v_2,\dots,\v_n)$表示为:

	$$ \span(\v_1,\v_2,\dots,\v_n) = \left\{ \sum_{i=1}^{n} a_k \v_k : \forall a_k \in \R \right\} $$
\defe

\egb
例如:

\begin{itemize}

\item  $(6,0,0)$和$(0,12,0)$的span,即$\span((6,0,0),(0,12,0))$是在$\R^3$中的一个平面,即所有$z=0$的点的集合。


\item $f_1(t) = t^2$和$f_2(t) = t^3$的span,即$\span(f_1,f_2)$是在$P(3)$中的一个子空间,即所有次数只有二次项和三次项的多项式的集合。

\end{itemize}

\ege 


\end{document}